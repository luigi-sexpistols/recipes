\clearpage
\section{Ashley's Mexican-inspired slow braised beef}

\index{meat!beef}
\index{meat!beef!brisket}
\index{starch!rice}
\index{author!Ashley Richmond}
\index{country!Mexico}

\subsection{Ingredients}

\begin{itemize}
	\item 1.5kg beef brisket, chunkily diced
	\item 2 onions, finely diced
	\item 1 capsicum, finely diced
	\item 4 cloves garlic, roughly minced
	\item 1 cup chicken stock
	\item 1 cup water
	\item \sfrac{1}{3} cup tomato paste
	\item \sfrac{1}{2} cup vinegar
	\item 1 dried Guajillo chilli, halved lengthways and seeds and veins removed
	\item 1 dried Ancho chilli, same treatment
	\item 4 anchovy fillets, mashed to paste, plus 1-2 tbsp of the oil
	\item 2 tsp smoked paprika
	\item 1\sfrac{1}{2} tsp fennel seeds
	\item 4 sprigs coriander, fresh, roughly minced
	\item Long-grain rice, for serving
\end{itemize}

Any kind of fairly neutral vinegar is suitable (e.g. not balsamic or anything more flavourful).

\subsection{Method}

As long as your oven heats up reasonably quickly, no pre-heating is needed due to the long cooking time.

\begin{enumerate}
	\item Generously salt the beef and leave uncovered in the fridge to dry up the surface moisture for at least 2 hours.
	\item Mix all the other ingredients in a bowl.
	\item Put the meat in a low, wide dish (a roasting pan or lasagne dish is ideal) and pour over the sauce. Wiggle the chunks of beef to ensure they are all coated.
	\item Bake in a \celsius{130} oven for 4 hours, or until the beef is extremely tender; falling apart is the goal. Stir every now and then to ensure nothing burns.
	\item Serve with rice, or however you like, I'm not the boss of you.
\end{enumerate}
