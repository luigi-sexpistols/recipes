\clearpage
\section{Braised Eggplant with Tofu in Garlic Sauce}

\index{author!J. Kenji Lopez-Alt}
\index{source!seriouseats.com}
\index{vegetarian}
\index{starch!rice}
\index{country!China}
\index{quick-n-easy}

Source: https://www.seriouseats.com/braised-eggplant-with-tofu-in-garlic-sauce-recipe

\subsection{Ingredients}

\begin{itemize}
    \item 500g eggplant, cut into 2" chunks
    \item 2 tsp Chinkiang vinegar
    \item \sfrac{3}{4} cup Shaoxing wine or dry sherry
    \item 1 tbsp cornstarch
    \item 3 tbsp soy sauce
    \item 2 tbsp brown sugar
    \item 1 tbsp doubanjiang paste
    \item 1 tbsp sesame oil
    \item 2 tbsp vegetable oil
    \item 2 cloves garlic, whole and smashed with the back of a knife
    \item 4 cloves garlic, sliced thinly
    \item 2 scallions, whites and greens thinly sliced and reserved separately
    \item 2-3 tbsp preserved mustard root, minced
    \item 350g firm silken tofu, cut into 1" chunks
    \item 2-3 tbsp fresh coriander leaves, roughly chopped
    \item Noodles or medium-grain rice, for serving.
\end{itemize}

Chinkiang vinegar is a black vinegar that can be found at most Asian supermarkets. Rice vinegar can be used in its place.

Preserved mustard root can be found in the canned or jarred section of an Asian market, sometimes labeled preserved or pickled Sichuan vegetable (check the ingredients label for mustard root).

Many folks like to purge their eggplants by salting them and letting them rest before cooking them. This step isn't necessary, particularly not when using Chinese or Japanese eggplants, which basically have no bitterness whatsoever and thus don't need to be purged (even modern cultivars of globe eggplants have very little bitterness).

\subsection{Method}

\begin{enumerate}
    \item Place eggplant in a large bamboo steamer and set over a wok filled with 2 inches of water. Bring to a boil over high heat, reduce to a simmer, cover steamer, and cook until eggplant is completely tender, about 10 minutes. Set aside.
    \item While eggplant is cooking, make the sauce. Combine vinegar, wine, and cornstarch and stir with fork until cornstarch is dissolved. Add soy sauce, brown sugar, chile paste, and sesame oil. Set aside.
    \item Wipe out wok and dry carefully. Add oil and whole garlic cloves. Heat over medium heat stirring and turning garlic cloves occasionally until light golden brown and fragrant, about 5 minutes. Discard whole cloves then increase heat to high and heat oil until smoking. Add sliced garlic, scallion whites, and preserved mustard root. Cook, stirring and tossing constantly until fragrant and just beginning to brown, about 1 minute. Stir sauce to re-incorporate corn starch, then add to wok, stirring constantly. Add eggplant and tofu and fold gently to combine. Bring to a boil, reduce to a simmer, and cook, folding and stirring occasionally until thick and glossy, about 5 minutes longer. Stir in scallion greens and cilantro and serve immediately with white rice, if desired.
\end{enumerate}
