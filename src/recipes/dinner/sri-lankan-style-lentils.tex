\section{Paripoo / \emph{Lentils, Sri Lankan-style}}

\index{starch!rice}
\index{vegetarian}
\index{author!Charmaine Solomon}
\index{country!Sri Lanka}
\index{curry!Sri Lankan}

From: Charmaine Solomon's Complete Asian Cookbook

Serves: 6

\subsection{Ingredients}

\begin{itemize}
	\item 500g red lentils
	\item 165mL coconut milk, plus 335mL water ("thin coconut milk")
	\item 1 dried red chilli, chopped
	\item 2 tsp dried prawn powder
	\item 1 tsp ground turmeric
	\item 1 tbsp ghee
	\item 6 curry leaves
	\item 2 onions, thinly sliced
	\item 20cm piece pandanus leaf
	\item 1 cinnamon stick
	\item 1 stem lemongrass
	\item 85mL coconut milk, plus 45mL water ("thick coconut milk")
	\item Long-grain rice, for serving
\end{itemize}

\subsection{Method}

\begin{enumerate}
    \item Wash the lentils well, removing any that float to the surface, then drain. 
    \item Place in a saucepan with the thin coconut milk, chilli, prawn powder, and turmeric.
    \item Bring to the boil, then reduce the heat to low, cover, and simmer until the lentils are just tender. 
    \item Heat the ghee in a separate saucepan over low heat. Add the curry leaves, onion, pandanus leaf, cinnamon stick, and lemongrass, and cook until the onion is brown.
    \item Remove half of the onion to a plate to use as a garnish.
    \item Add the remaining onion to the lentil mixture, along with the thick coconut milk, and stir to combine.
    \item Season with salt and simmer until the lentils are very soft and the consistency of runny porridge.
    \item Serve with rice and curries.
\end{enumerate}
