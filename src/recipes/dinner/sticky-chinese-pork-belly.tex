\section{Sticky Chinese Pork Belly}

\index{starch!rice}
\index{meat!pork}
\index{meat!pork!belly}
\index{country!China}
\index{source!kitchensanctuary.com}
\index{author!Nicky Corbishley}

From https://www.kitchensanctuary.com/sticky-chinese-pork-belly/

\subsection{Ingredients}

\begin{itemize}
	\item 1kb pork belly, rindless, cut into 2\sfrac{1}{2}x5cm pieces
	\item 1L chicken stock, hot
	\item 5cm piece ginger, minced
	\item 3 cloves garlic, roughly chopped
	\item 1 tbsp Chinese cooking wine
	\item 1 tbsp sugar
	\item 2 tbsp vegetable oil
	\item 1 red chilli, finely chopped
	\item 2 tbsp honey
	\item 2 tbsp brown sugar
	\item 3 tbsp dark soy sauce
	\item 1 stem lemongrass
\end{itemize}

Optionally, add in carrots, onions, and other vegetables before the pork at the frying stage.

\subsection{Method}

\begin{enumerate}
	\item Add pork, stock, half the ginger, garlic, cooking wine, and sugar to a pot and bring to a boil.
	\item Cover, then reduce heat and simmer for two hours.
	\item While the pork is cooking, mix half the oil, salt, pepper, remaining ginger, chilli, honey, brown sugar, soy sauce, and lemongrass. Set aside for now; this will be the glaze.
	\item Turn off the heat, remove the pork, and pat it dry. Reserve the liquid for noodle soup at a later time.
	\item Chop pork into bite-sized chunks.
	\item In a frying pan over medium-high heat, add the pork, a pinch of salt and pepper, and remaining oil and fry until the pork starts to turn golden.
	\item Pour the glaze mixture over the pork and continue to cook for a few minutes, until the pork looks dark and sticky.
	\item Serve with rice.
\end{enumerate}
