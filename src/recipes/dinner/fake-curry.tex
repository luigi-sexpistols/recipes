\section{Fake Curry}

\index{meat!beef}
\index{meat!lamb}
\index{starch!rice}
\index{curry!Indian}
\index{curry!fake}
\index{source!Clare Richmond}

  \begin{multicols}{2}
    \subsection{Ingredients}
      \begin{itemize}
        \item Ghee
        \item 1 large onion, sliced thickly
        \item 3 cloves garlic, crushed
        \item 1 stick celery, thinly sliced
        \item 500g beef or lamb, cubed
        \item 1 tbsp garam masala
        \item \sfrac{1}{2} tsp ground tumeric
        \item \sfrac{1}{2} tsp ground cumin
        \item \sfrac{1}{3} tsp ground cardamom
        \item 1 tsp powdered ginger
        \item 4-5 cury leaves
        \item 1 tsp dried chilli flakes
        \item \sfrac{1}{2} can tomatoes (crushed or diced)
        \item \sfrac{2}{3} cup water
        \item 2-5 tablespoons natural yoghurt
      \end{itemize}
  \vfill\null
  \columnbreak
  \subsection{Method}
    \begin{enumerate}
      \item On a low heat, melt enough ghee to cover the bottom of the pot, and add the onion, garlic, and celery.
      \item Cook until the onion is quite soft, stirring occasionally.
      \item Increase the heat to medium and add the meat.
      \item Allow the meat to brown, then add the spices -- garam masala, turmeric, cumin, cardamom, curry leaves, and chilli. Salt and pepper to taste.
      \item Cook for a few minutes, then add the tomatos and water.
      \item Simmer over low to medium heat for 2 hours, stirring every 20-30 minutes.
      \item Stir the yoghurt through, then serve with long grain rice.
    \end{enumerate}
  \end{multicols}
