\section{Kenji's Ragu Bolognese}

\index{author!J. Kenji Lopez-Alt}
\index{meat!beef}
\index{meat!beef!mince}
\index{meat!lamb!mince}
\index{meat!lamb}
\index{meat!pork!mince}
\index{meat!pork}
\index{starch!pasta}
\index{country!Italy}

\begin{quote}
    Ragu Bolognese is easy, actually.
\end{quote}

Source: https://www.youtube.com/watch?v=cvROmO5ODnQ

\subsection{Ingredients}

\begin{itemize}
    \item A glug of extra-virgin olive oil
    \item 160g finely diced pancetta or salt pork
    \item 250g each of ground beef, lamb, and pork (or any combination that includes beef)
    \item Salt and pepper
    \item 1 large onion, minced
    \item 2 large ribs of celery, minced
    \item 1 large carrot, minced
    \item 4-5 cloves of fresh garlic, smashed and minced
    \item Handful of minced fresh parsley and sage (optional, you can also use a bundle of rosemary or thyme)
    \item \sfrac{1}{4} cup tomato paste
    \item 1 cup dry wine (white or red)
    \item 2 cups chicken stock
    \item 1 cup milk
\end{itemize}
    
If using store-bought stock, sprinkle it with a couple tablespoons of gelatin and set it aside until the gelatin is hydrated before using it.

\subsection{Method}

\begin{enumerate}
    \item Heat the oil in a wide straight-sided sauté pan or Dutch oven over medium-high heat until shimmering. Add the pancetta and cook, stirring occasoinally, until it's well-browned and the fat has mostly rendered off, a few minutes. Add the meat, season lightly with salt and pepper, and cook, breaking it up with a spoon, until it's pretty well-browned as well, 7 minutes or so.
    \item Add the onion, celery, carrot, and garlic and cook, stirring, until the vegetables are softened but not browned, about 5 minutes. Add the minced parsley and/or sage (or the bundle of rosemary or thyme) and the tomato paste and cook, stirring, until fragrant. There should be a large amount of browned solids on the bottom of the pan by now.
    \item Add the wine and cook, scraping up the browned bits with a wooden spoon. Continue cooking until sauce is thick and the wine has fully reduced. Add the stock and milk and stir to combine.
    \item Bring to a boil, then reduce heat until the sauce is at the barest simmer. Cook, stirring now and then, until the sauce is rich and thick and emulsified. Fat may break out as it cooks and form a reddish slick on top. This is OK, just keep cooking and that fat will eventually get re-emulsified into the sauce.
    \item To serve, cook up some good fresh pasta (tagliatelle is traditional) or a hearty bronze-cut dry pasta such as rigatoni just until al dente (typically 10\% or so less than package directions) in water that is about as salty \emph{as your tears}. Transfer the cooked pasta to a large skillet and spoon some of the ragu on top of it. Add some minced parsley and/or basil, some freshly grated cheese, a drizzle of olive oil, and a big splash of the pasta cooking-water. Cook over high heat, tossing constantly, until the sauce is creamy and coats the pasta (add more pasta water as necessary if it looks dry or if the fat breaks out). Serve right away.
\end{enumerate}
