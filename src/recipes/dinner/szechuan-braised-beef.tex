\section{Szechuan Braised Beef}

\index{meat!beef}
\index{meat!beef!brisket}
\index{starch!potato}
\index{starch!rice}
\index{country!China}

\subsection{Ingredients}

\begin{itemize}
  \item 1kg beef brisket (or other such cut)
  \item 4 medium potatoes
  \item 2 tbsp Szechuan doubanjiang
  \item 2 leeks
  \item 1 clove garlic
  \item 1 inch piece of ginger
  \item 1 tbsp rock suger
  \item 2 star anise pods
  \item 2 bay leaves
  \item 1 small piece Cassia bark
  \item \sfrac{1}{2} tbsp Szechuan peppercorns
  \item \sfrac{1}{4} cup light soy sauce
  \item 2 tbsp dark soy sauce
  \item 4 tbsp Chinese cooking wine
  \item Medium-grain rice, for serving (optional)
\end{itemize}

\subsection{Method}

\begin{enumerate}
  \item Cut the meat into small chunks and soak in a large pot of clear water for 30 minutes.
  \item Divide the ginger in half, lightly crushing one half and slicing the other.
  \item To a pot, add two slices of ginger, 1 tbsp of the cooking wine, and the meat, cover with water, then light the burner. Bring to a boil for 2 minutes, then remove the meat from the water and drain.
  \item Over medium heat, fry the doubanjiang in 1 tbsp of oil until the oil turns red. Add the leeks, garlic, ginger, and other spices and fry until aromatic.
  \item Add the meat. Once it has gained some colour, add the sugar, light soy sauce, and dark soy sauce, and continue to cook for a couple of minutes.
  \item Add water to cover the meat, plus an additiional 3cm. Cook until the meat is soft.
  \item Peel the potatoes and cut into 3cm cubes. Boil in a separate pot until almost soft (around 20 minutes). Drain and add to the main dish, cooking for another 2 minutes.
\end{enumerate}

