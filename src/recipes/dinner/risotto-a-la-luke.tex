\section{Risotto}

\index{meat!chicken}
\index{meat!chicken!thigh}
\index{meat!pork}
\index{meat!fish}
\index{meat!fish!salmon}
\index{vegetarian}
\index{starch!rice}
\index{country!Italy}
\index{quick-n-easy}

\begin{multicols}{2}
  \subsection{Ingredients}
    \begin{itemize}
      \item 50g butter
      \item 20mL olive oil
      \item 2 small shallots, diced fine
      \item 2 cloves garlic, minced
      \item 400g Arborio rice
      \item 1.2L cooking liquid (see right)
      \item Parmesan
    \end{itemize}
  \subsection{Mixin Options}
    \subsubsection{Chicken and Mushroom}
      Saute brown mushrooms, reconstituted porcini, and diced chicken thighs. Add to risotto at step 5 with minced parsley. Cooking liquid is chicken stock, the water used to reconstitute the porcini and white wine.
    \subsubsection{Salmon}
      Saute cubed atlantic salmon. Add to risotto at step 5 with drained diced canned tomatoes and coriander. Cooking liquid is fish or vegetable stock, tomato juice from canned tomatoes and dry vermouth, steeped with a few threads of saffron.
    \subsubsection{Bolognese}
      Make a bolognese sauce, and marble it through the risotto at the mixins stage. Cooking liquid is vegetable stock and red wine.
    \subsubsection{Pork}
      Saute pork and diced granny smith apples. Cooking liquid is chicken stock and apple cider.
    \subsubsection{Beef Sausages}
      Fry onion (start first) and sausages (add halfway through) until nicely brown, add to rice with frozen peas. Cooking liquid is beef stock and red wine.
  \vfill\null
  \columnbreak
  \subsection{Cooking Liquid}
  The cooking liquid should match your mixins, generally should be about 80:20 stock-to-wine ratio. For plain risotto, use 250mL white wine and 950mL chicken stock.

  \subsection{Method}
    \begin{enumerate}
      \item Combine rice and cooking liquid in a large vessel, agitating the rice until liquid is cloudy. Drain rice through a sieve, allowing the cooking liquid to collect into another vessel. Let the rice drain in the sieve for five to ten minutes. Stir the cooking liquid to redistribute the starch, seperate one cup of the cooking liquid and reserve.
      \item Melt half the butter and olive oil in a large flat bottomed skillet with a lid until foaming subsides. Fry the shallot on medium high heat until translucent, and then add all the rice, cooking and tossing until rice is toasted, translucent on the edges with a nutty aroma (around 5 minutes). Add the garlic and toast for an additional minute.
      \item Add the large portion of the cooking liquid to the rice, stir and cover. Reduce heat to lowest, and cook for about ten minutes without disturbing.
      \item Stir the rice and cooking liquid mixture, shaking to even out rice layer. Cover, and cook until rice is 90\% done, with just a little bite when tasted (start tasting at 8 minutes, can take up to 10).
      \item Increase temperature to high and add the last cup of cooking liquid (stir beforehand to redistribute starch into liquid) and remaining butter. Add in mixins. Stir and toss constantly until creamy and thick, just a touch thinner than your serving consistency (can take up to 5 minutes, don’t worry if it looks too runny, just keep cooking). Remove from the heat, add cheese, and serve.
    \end{enumerate}
  \end{multicols}
