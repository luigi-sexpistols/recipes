\section{Dal / \emph{Lentil pur\'{e}e}}

\index{starch!rice}
\index{starch!bread}
\index{vegetarian}
\index{author!Charmaine Solomon}
\index{country!India}
\index{curry}

From: Charmaine Solomon's Complete Asian Cookbook

Serves: 4

\subsection{Ingredients}

\begin{itemize}
	\item 250g red lentils
	\item 1\sfrac{1}{2} tbsp ghee
	\item 1-2 large onions, thinly sliced
	\item 2 cloves garlic, minced
	\item 1 tsp ginger, finely grated, fresh
	\item \sfrac{1}{2} tsp ground turmeric
	\item 1 tsp salt
	\item \sfrac{1}{2} tsp garam masala
\end{itemize}

Instead of the garam masala, I used a mix of cardamom, coriander, and cloves (all ground), totalling about 1\sfrac{1}{2} tsp, which was quite nice.

This dish can be garnished with fried onions; start this at the same time as the onions for the main dish, and just keep going until the end.

\subsection{Method}

\begin{enumerate}
	\item Wash the lentils thoroughly, removing and discarding those that float on the surface. Drain well.
	\item Heat the ghee in a saucepan over low heat. Add the onion, garlic, and ginger, and cook until the onion is golden brown (about 30 minutes?).
	\item Add the turmeric and stir to coat the onion. Add the drained lentils and cook for 1-2 minutes, then add 750mL hot water. Bring to a boil, then reduce the heat to low, cover, and simmer for 15-20 minutes or until the lentils are "just tender".
	\item Add the salt and spices, mix well, and continue cooking until the lentils are soft and the consistency is similar to porridge.
	\item Serve with rice, Indian breads, or as a light meal by itself.
\end{enumerate}
