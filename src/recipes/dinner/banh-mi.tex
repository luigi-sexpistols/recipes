\section{Banh Mi}

\index{author!Nagi Maehashi}
\index{source!recipetineats.com}
\index{meat!chicken}
\index{meat!chicken!breast}
\index{starch!bread}
\index{country!Vietnam}

Source: https://www.recipetineats.com/chicken-banh-mi-vietnamese-sandwich/

\subsection{Ingredients}

\begin{multicols}{2}
    Pickled vegetables

    \begin{itemize}
        \item 2 medium carrots, peeled and cut into matchsticks
        \item \sfrac{1}{2} large white radish (daikon), peeled, cut into matchsticks
        \item 1\sfrac{1}{2} cups boiling water
        \item \sfrac{1}{2} cup white sugar
        \item 4 tsp salt
        \item \sfrac{3}{4} cup rice wine vinegar
    \end{itemize}

    \vfill\null
    \columnbreak

    Banh mi

    \begin{itemize}
        \item 400g meat of your choice
        \item 4 crusty white rolls
        \item 2-3 tbsp salted butter
        \item Chicken pate
        \item 4 tbsp whole egg mayonnaise (Kewpie preferred)
        \item 2 green onion stems, cut to matchstick length
        \item 1 cucumber, thinly sliced lengthwise
        \item 1\sfrac{1}{2} cups cilantro sprigs
        \item Maggi seasoning
        \item 2-3 birds eye or Thai red chilli, finely sliced
    \end{itemize}
\end{multicols}

\subsection{Method}

\begin{enumerate}
    \item In a large bowl, dissolve the salt and sugar in the hot water. Stir in vinegar. Add carrots and daikon; they should just about be covered. Leave for 2 hours until slightly floppy. Drain well before using.
    \item Split the bread roll in half but don't cut all the way through.
    \item Spread the lid with butter. Spread the base with 1-2 tbsp pate (be generous!), then 1\sfrac{1}{2} tbsp mayonnaise.
    \item Pile on the meat, drizzle with \sfrac{1}{4} tsp Maggi seasoning (you can always add more later).
    \item Stuff with pickled vegetables, 2 slices of cucumber, 3 slices of green onion, coriander, and chilli to taste.

\end{enumerate}
