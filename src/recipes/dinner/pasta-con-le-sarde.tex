\section{Pasta con le Sarde / \emph{Sicilian Pasta With Sardines}}

\index{author!Daniel Gritzer}
\index{source!seriouseats.com}
\index{meat!fish}
\index{meat!fish!sardine}
\index{starch!pasta}
\index{country!Italy}

Source: https://www.seriouseats.com/pasta-pasta-con-le-sarde-sicilian-pasta-with-sardines

\subsection{Ingredients}

\begin{itemize}
    \item \sfrac{1}{2} cup dry white wine
    \item \sfrac{1}{4} cup golden raisins
    \item 4\sfrac{1}{2} tbsp extra-virgin olive oil, divided, plus more as needed
    \item \sfrac{1}{2} tsp ground fennel seed
    \item \sfrac{1}{2} cup panko bread crumbs
    \item 400g fennel bulb, peeled, cored, and minced, fronds reserved
    \item 1 small yellow onion
    \item 4 anchovy fillets (oil-packed)
    \item \sfrac{1}{4} cup pine nuts, toasted
    \item 350g canned sardines
    \item Long pasta (e.g. bucatini or spaghetti), for serving
\end{itemize}

\subsection{Method}

\begin{enumerate}
    \item In a microwave-safe heatproof bowl, heat wine until steaming (alternatively, heat wine on the stovetop in a small saucepan). Add raisins and saffron, if using. Set aside.
    \item In a small skillet, heat 1\sfrac{1}{2} tbsp olive oil over medium-high heat until shimmering. Add ground fennel and cook until fragrant, about 20 seconds. Add bread crumbs and cook, tossing, until lightly toasted. Transfer breadcrumbs to a bowl and season well with salt and pepper. Set aside.
    \item Bring a large pot of salted water to a boil. Mince half the fennel fronds and reserve the other half (uncut).
    \item In a large skillet or sauté pan, heat remaining 3 tbsp olive oil over medium heat until shimmering. Add diced onion and diced fennel bulb and cook, stirring, until onion and fennel are soft and tender, about 8 minutes. Add anchovy fillets and cook, stirring, until dissolved in the oil.
    \item Add wine, raisins, and saffron, and cook, scraping up any browned bits from bottom of pan, until wine is almost entirely evaporated.
    \item Add pine nuts and sardines and cook, stirring, until sardines are just barely cooked through, about 2 minutes. Remove from heat.
    \item Meanwhile, cook the pasta until al dente, then drain, reserving at least 1 cup pasta cooking water.
    \item Transfer pasta to skillet along with \sfrac{1}{4} cup pasta cooking water. Return skillet to medium-high heat and cook, stirring and tossing, until pasta is well-coated in sauce and any excess liquid has cooked off. Drizzle on some fresh olive oil (don't be shy) along with the minced fennel fronds, and toss well. Season with salt. Add a very small handful of bread crumbs and toss once more.
    \item If at any point the pasta becomes too dry, add additional pasta cooking water in \sfrac{1}{4}-cup increments, and toss to loosen and moisten (you can also drizzle on more olive oil as desired). The noodles should be slick and glossy with a sheen of sauce, but not sitting in a watery puddle.    
    \item Divide pasta into serving bowls, making sure to distribute sardines, pine nuts, and raisins evenly. Top with a more generous handful of breadcrumbs. Garnish with whole fennel fronds and serve right away.
\end{enumerate}
