\clearpage
\section{Hao SiFu's Pork Bone Broth}
  \begin{multicols}{2}
    \subsection{Ingredients}
      \begin{itemize}
        \item 2-3kg pork leg bones and knuckles
        \item 500g chicken bones (neck or wings work best)
        \item 2 large onions, halved
        \item 6 cloves garlic, lightly crushed
        \item \sfrac{1}{2} Chinese white radish
        \item 1 thumb-sized piece of ginger
        \item 100g dried shitake mushrooms, soaked
        \item 100g conpoy (dried scallop)
        \item 1 tsp crushed black pepper
        \item Several dashes Chinese rice wine
      \end{itemize}
      If you cannot find Chinese white radish, you can use carrot instead.

      This broth is served with fresh noodles with vegetables and protein of your choice. Ensure you season with soy sauce or salt before serving. For example, serve with roast pork, bok choy, and mushrooms with egg noodles.
  \vfill\null
  \columnbreak
  \subsection{Method}
    \begin{enumerate}
      \item Boil the bones in a separate pot for at least 10 minutes. Remove from water and rinse carefully under tap water.
      \item Crack the leg bones along the middle (use a giant cleaver) Place washed and cracked bones in clean pot with cold water and bring to boil.
      \item Add all other ingredients.
      \item Gently boil with lid slightly ajar for at least 8 hours (the longer the better) stirring occasionally. It will smell terrible for the first 2 hours for some reason – this is normal. I've also used pressure cooker which reduces the cooking time to about 2 hours. It's still very good but unfortunately you don't get the nice emulsification of the fats.
      \item The finished product will look like a rich, opaque, creamy white colour. Strain before serving.
    \end{enumerate}
  \end{multicols}
