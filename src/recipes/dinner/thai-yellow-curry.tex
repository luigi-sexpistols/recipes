\section{Thai Yellow Curry}

\index{author!Nagi Maehashi}
\index{meat!seafood}
\index{meat!seafood!prawns}
\index{starch!rice}
\index{country!Thailand}
\index{curry!Thai}
  
  Source: https://www.recipetineats.com/thai-yellow-curry/
    
  \subsection{Ingredients}
    
  Curry Paste:
  
  \begin{itemize}
    \item 10 dried red chillis, chopped into 1cm pieces
    \item 1-4 fresh birds eye chillis, deseeded, roughly (1 for mild, 4 for spicy)
    \item 2 lemongrass stems
    \item 1-2 eschalots, roughly chopped
    \item 2 tbsp fresh turmeric, finely grated \emph{(USE GLOVES)} \emph{or} 1 \sfrac{1}{2} tsp ground
    \item 2 tbsp galangal, finely grated
    \item 8 cloves garlic, roughly chopped
    \item 1\sfrac{1}{2} tbsp Thai shrimp paste in bean oil
    \item 1 tsp ground coriander
    \item 1 tsp ground cumin
    \item \sfrac{1}{4} tsp ground cardamom
    \item \sfrac{1}{2} tsp fenugreek powder
    \item \sfrac{1}{8} tsp white pepper
  \end{itemize}

  Curry:

  \begin{itemize}
    \item 3 tbsp vegetable oil
    \item 1 medium potato, peeled, cut into 2.5x1cm pieces
    \item 1 small carrot, peeled, sliced into 0.5cm slices diagonally
    \item 1 cup low sodium chicken stock
    \item 300mL 100\% coconut cream
    \item 4 tsp fish sauce
    \item 5 tsp white sugar
    \item 2 tsp tamarind puree
    \item 350g unpeeled prawns/shrimp, medium, peeled
    \item \sfrac{1}{2} cup bamboo shoots
  \end{itemize}

  All of the chillis are optional.

  From the author: \emph{
    Thai shrimp paste in bean oil - I use Por Kwan brand, the most popular one sold at Asian grocery stores here in Australia. [...] If you can't find Shrimp Paste in oil, Belacan is a substitute that's nearly as good [...]. Use 1.5 tbsp, roughly chop then toast on low heat in 1 tbsp oil for 3 minutes. Then use in place of shrimp paste.
  }
  
  \subsection{Method}

  Curry Paste:
  
  \begin{enumerate}
    \item Roughly chop chillies and transfer to bowl, leaving behind seeds. Cover with boiling water and soak for 30 minutes then drain (reserve soaking water).
    \item Check spiciness: Have a nibble of soaked chilli, should not be that spicy. If it is spicy, only use \sfrac{1}{3} to \sfrac{1}{2} of the amount.
    \item Remove woody top half and outer layers of lemongrass. Grate with microplane.
    \item Place chillis, lemongrass and all remaining curry paste ingredients in a jar just wide enough to fit a stick blender. Add 3 tablespoons chilli soaking water. Blitz with stick blender until smooth so there's no hard grit - rub between your fingers to check - about 15 seconds on high (or use small food processor, scraping down sides well).
  \end{enumerate}

  Curry:

  \begin{enumerate}
    \item Cook off curry paste: Heat oil in a medium heavy based skillet over medium heat. Add curry paste and cook for 3 to 4 minutes until it dries out a bit and smells fragrant.
    \item Add chicken stock, stir to dissolve paste, then simmer for 1 minute.
    \item Reduce heat to medium low. Add tamarind, fish sauce and sugar. Stir until tamarind is dissolved. \item Stir in coconut, carrot and potato.
    \item Bring to simmer, then simmer gently for 15 minutes or until potato is almost fully soft. Pierce with knife to check, it might take 20 minutes.
    \item Add prawns and bamboo shoots. Stir, then cook for 3 minutes until prawns are just cooked.
    \item Taste and adjust the curry sauce at this point. Thin sauce with stock or water, add salt, fish sauce or sugar if needed.
    \item Serve! Garnish with Thai Basil, fresh chilli and crispy shallots. Serve with jasmine rice.
  \end{enumerate}
