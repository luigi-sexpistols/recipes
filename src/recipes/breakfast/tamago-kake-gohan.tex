\section{Tamago Kake Gohan}

  Tamago gohan (literally "egg rice") - rice mixed with a raw egg - is Japanese comfort food at its simplest. Start with a bowl of hot rice, then break an egg into it. Season it with a little bit of soy sauce, a pinch of salt, and a shake of Aji-no-moto, a Japanese brand of pure powdered MSG. (Like most Japanese people, I have no hang-ups about eating MSG.) Whip it up with a pair of chopsticks until the egg turns pale yellow and foamy and holds the rice in a light, frothy suspension, somewhere between a custard and a meringue.

  \begin{multicols}{2}
    \subsection{Ingredients}
      \begin{itemize}
        \item 1 cup cooked hot white rice
        \item 1 egg
        \item 1 egg yolk (optional)
        \item \sfrac{1}{2} tsp Soy sauce
        \item 1 pinch salt
        \item 1 pinch MSG powder (optional)
        \item \sfrac{1}{2} tsp mirin (optional)
        \item 1 pinch Hondashi (optional)
        \item Furikake (optional)
        \item Nori, thinly sliced (optional)
      \end{itemize}
      Hondashi is powdered dashi that can be found in any Japanese market and most well-stocked supermarkets. Furikake is a seasoning mixture typically made with seaweed, dried sweetened bonito, and sesame seeds, among other ingredients. It can be found in any Japanese market.
  \vfill\null
  \columnbreak
  \subsection{Method}
    \begin{enumerate}
      \item Place rice in a bowl and make a shallow indentation in the centre.
      \item Break the whole egg into the hollow and season with soy sauce, salt, MSG powder, mirin, and Hondashi.
      \item Stir vigorously with chopsticks to incorporate egg: it should become pale yellow, frothy, and fluffy in texture.
      \item Taste, and adjust seasoning as necessary.
      \item Optionally, sprinkle with furikake and nori and add the other egg yolk into a hollow on top of the rice.
    \end{enumerate}
  \end{multicols}
